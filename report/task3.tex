\section{Задание 3}

\begin{enumerate}
	\item Построить датчик экспоненциального распределения. Проверить для 
    данного распределения свойство отсутствия памяти. Пусть 
    $X_1,X_2,\dots,X_n$~--- независимые экспоненциально распределенные с.в. с 
    параметрами $\lambda_1, \lambda_2, \dots, \lambda_n$ соответственно. Найти 
    распределение случайной величины $Y = \min(X_1, X_2, \dots, X_n)$.
	\item На основе датчика экспоненциального распределения построить датчик 
    пуассоновского распределения.
	\item Построить датчик пуассоновского распределения как предел биномиального
     распределения. С помощью критерия хи-квадрат Пирсона убедиться, что получен
     датчик распределения Пуассона.
	\item Построить датчик стандартного нормального распределения методом 
    моделирования случайных величин парами с переходом в полярные координаты. 
    Проверить при помощи $t$-критерия Стьюдента равенство математических 
    ожиданий, а при помощи критерия Фишера равенство дисперсий.  
\end{enumerate}

\subsection{Экспоненциальное распределение}
    Напомним, что $\eta \sim \mathrm{Uni}\inter{0}{1}$, тогда для $\epsilon \sim
    \mathrm{Exp}(\lambda)$ справделиво 
    \footnote{см. \cite{NMS} гл. 4 \S 1 ``Метод обратной функции''.} 
    \[\epsilon \sim -\frac{1}{\lambda} \ln\eta.\]
    Результат моделирования см. рис. \ref{task3_exp}

    Экспоненциальное распределение обладает свойством отсутствия памяти 
    (рис. \ref{task3_memoexp}), т.е. аналогично \S\,\ref{par12}:
    \[\bigl. \epsilon_m \sim \epsilon ,\quad \forall{m} \in \mathbb{N} \cup \{0\},\]
    где $\epsilon_m := (\gamma \bigr|_{\Omega_m} - m),\: 
    \Omega_m = \epsilon^{-1}(\epsilon \geqslant m) \in \mathcal{A}.$ 
    \bigskip

    \begin{figure}[tbp]
        \centering
        \begin{subfigure}[b]{0.4\textwidth}
            \centering
            \includegraphics[width=\textwidth]{resources/task3_exp.png}
            \caption{Экспоненциальное распределение}
            \label{task3_exp}
        \end{subfigure}
        \hfill
        \begin{subfigure}[b]{0.5\textwidth}
            \centering
            \includegraphics[width=\textwidth]{resources/task3_memoexp.png}
            \caption{Отсутствие памяти ($m = 2$)}
            \label{task3_memoexp}
        \end{subfigure}
        \caption{}
    \end{figure}

    Распределение сл.в. $Y = \min(X_1,X_2,\ldots X_n)$ имеет функцию 
    распределения 
    \begin{multline*}
        F(y) = \Prb{Y < y} = 1 - \Prb{Y \ge 1} = 
        1 - \Prb{\min(X_1,X_2,\ldots X_n) \ge 1} = \\
        1 - \Prb{X_1 \ge y, \ldots, X_n \ge y} = 
        \text{\{в силу независимости сл.в. $X_i$\} } = 
        1 - \prod_{i=1}^{n} \Prb{X_i \ge y} = \\
        1 - \prod_{i=1}^{n} (1 - F_{Exp(\lambda_i)}) = 
        1 - \prod_{i=1}^{n} (1 - (1 - e^{-\lambda_i y})) =
        1 - \prod_{i=1}^{n} e^{-\lambda_i y} = 
        1 - e^{-(\sum_{i=1}^{n} \lambda_i) y}.
    \end{multline*}
    Таким образом
    \[Y \sim \mathrm{Exp}(\sum_{i=1}^{n} \lambda_i)\]
    Cравните результаты, посчитанные для обоих представлений $Y$ 
    на рис. \ref{task3_minexp}.

    \begin{figure}[tbp]
        \centering
        \includegraphics[width=0.7\textwidth]{resources/task3_minexp.png}
        \caption{}
        \label{task3_minexp}
    \end{figure}

\subsection{Датчик пуассоновского распределения 1}
    Для $\pi \sim \mathrm{Pois}(\lambda)$ верно следующее представление
    \footnote{см. \cite{NMS} гл. 5 \S 1 ``Моделирование дискретных величин''.}
    \[\pi = \max_{\mathbb{N}\cup\{0\}}(n \mid S_n = \sum_{i=1}^{n} \epsilon_i \le 1),\]
    где $\epsilon_1,\epsilon_2,\ldots$~--- пуассоновские с параметром $\lambda$ 
    н.о.р.с.в. 

    Результат моделировния см. \ref{task3_pois1}

    \begin{figure}[tbp]
        \centering
        \includegraphics[width=0.5\textwidth]{resources/task3_pois1.png}
        \caption{}
        \label{task3_pois1}
    \end{figure}