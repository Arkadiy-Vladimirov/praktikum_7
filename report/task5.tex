\section{Задание 5}

\begin{enumerate}
	\item Пусть $X_i \sim \mathcal{N}(\mu, \sigma^2)$. Убедиться эмпирически в 
	справедливости ЗБЧ и ЦПТ, т.е. исследовать поведение суммы $S_n$ и 
	эмпирического распределения величины
	\begin{equation*}
		\frac{S_n - \mu n}{\sigma \sqrt{n}}.
	\end{equation*}
	\item Считая $\mu$ и $\sigma^2$ неизвестными, для пункта 1 построить 
	доверительные интервалы для среднего и дисперсии.
	\item Пусть $X_i \sim K(a, b)$ имеет распределение Коши со сдвигом $a$ и 
	масштабом $b$. Проверить эмпирически, как ведут себя суммы $S_n/n$. 
	Результат объяснить, а также найти закон распределения данных сумм.
\end{enumerate}

\subsection{Проверка ЗБЧ и ЦПТ}
	Действительно, для выборки $\{X_i\} \sim \mathcal{N}(\mu,\sigma^2)$ значение 
	$\dfrac{S_n}{n} = \dfrac{\sum_{i=1}^n X_i}{n}$ ведет себя в соответствии с 
	законом больших чисел (см. рис. \ref{LLN}), т.е
	\begin{equation*}
		\frac{S_n}{n} \rightarrow \mu \text{, при } n \rightarrow \infty.
	\end{equation*}

	Результаты опыта соответствуют и центральной пределной теореме, т.е
	\begin{equation*}
		\frac{S_n - \mu n}{\sigma \sqrt{n}} \xrightarrow{d} \eta \sim 
		\mathcal{N}(0,1) \text{ as } n \rightarrow \infty. 
		\text{ (см. рис. \ref{CLT})}
	\end{equation*}

	\begin{figure}[tbp]
        \centering
        \begin{subfigure}[b]{0.45\textwidth}
            \centering
            \includegraphics[width=\textwidth]{resources/task5_LLN.png}
            \caption{ЗБЧ ($\mu = -2,\:\sigma^2 = 0.5$)}
            \label{LLN}
        \end{subfigure}
        \hfill
        \begin{subfigure}[b]{0.5\textwidth}
            \centering
            \includegraphics[width=\textwidth]{resources/task5_CLT.png}
            \caption{ЦПТ ($\mu = -2,\:\sigma^2 = 0.5$)}
            \label{CLT}
        \end{subfigure}
        \caption{}
    \end{figure}
	
\subsection{Доверительные интервалы}



\subsection{Поведение сумм $S_n/n$ для распределения Коши}