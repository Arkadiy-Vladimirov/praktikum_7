\section{Задание 9}

Рассмотреть два вида процессов:
\begin{itemize}
	\item Винеровский процесс $W(t), t \in [0,1], W(0) = 0$.
	\item Процесс Орнштейна--Уленбека $X(t), t \in [0,1], X(0) = X_0$, то есть 
    стационарный марковский гауссовский процесс. Начальные значения $X_0$ 
    генерируются случайным образом так, чтобы полученный процесс был 
    стационарным.
\end{itemize}

Для данных гауссовских процессов
\begin{enumerate}
	\item Найти ковариационную функцию и переходные вероятности.
	\item Моделировать независимые траектории процесса с данными переходными 
    вероятностями методом добавления разбиения отрезка.
	\item Построить график траектории, не соединяя точки ломаной, с целью 
    получения визуально непрерывной линии.
\end{enumerate}
