\section{Задание 2}
    \begin{enumerate}
        \item Построить датчик сингулярного распределения, имеющий в качестве 
        функции распределения канторову лесницу. С помощью критерия Колмогорова 
        убедиться в корректности работы датчика.
        \item Для канторовых случайных величин проверить свойство симметричности 
        относительно $\frac{1}{2}$ ($X$ и $1 - X$ распределены одинаково) и 
        самоподобия относительно деления на 3 (условное распределение $Y$ при 
        условии $Y \in \segm{0}{\frac{1}{3}}$ совпадает с распределением 
        $Y/3$) с помощью критерия Смирнова.
        \item Вычислить значение математического ожидания и дисперсии для 
        данного распределения. Сравнить теоретические значения с эмпирическими 
        для разного объема выборок. Проиллюстрировать сходимость.
    \end{enumerate}

    \subsection{Датчик для канторова распределения}
        Носителем канторова распределения является счетное пересечение множеств

        \begin{align*}
            C_0 = {} & [0,1] \\
            C_1 = {} & [0,1/3] \cup [2/3,1] \\
            C_2 = {} & [0,1/9] \cup [2/9,1/3] \cup [2/3,7/9] \cup [8/9,1] \\
            C_3 = {} & [0,1/27] \cup [2/27,1/9] \cup [2/9,7/27] \cup [8/27,1/3] \cup \\
                  {} & [2/3,19/27] \cup [20/27,7/9] \cup [8/9,25/27] \cup [26/27,1] \\
            C_4 = {} & \cdots,
        \end{align*}

        что дает нам естественный способ рекурсивного выражения сл.в. 
        $\delta \sim \mathrm{Cant}$ через $\xi \sim \mathrm{Bern(0.5):}$

        \begin{gather*}
            \delta = \phi_0 \cdot \Ind{\xi = 0} + \phi_1 \cdot\Ind{\xi = 1} \\
            \phi_0 = \phi_{00} \cdot \Ind{\xi = 0} + 
                                            \phi_{01} \cdot\Ind{\xi = 1}, \quad
            \phi_1 = \phi_{10} \cdot \Ind{\xi = 0} + 
                                                \phi_{11} \cdot\Ind{\xi = 1}\\
            \cdots
        \end{gather*}

        \begin{gather*}
            \Prb{\phi_{i_1 i_2 \ldots i_n} \in 
                \segm{\sum_{k=1}^{n} \frac{2 i_k}{3^k}}
                {\:\sum_{k=1}^n \frac{2 i_k}{3^k} + \frac{1}{3^n}} } = 1,\\
            \phi_{i_1 i_2 \ldots i_n} = 
                \phi_{i_1 i_2 \ldots i_n 0} \cdot \Ind{\xi = 0} + 
                \phi_{i_1 i_2 \ldots i_n 1} \cdot \Ind{\xi = 1}.            
        \end{gather*}