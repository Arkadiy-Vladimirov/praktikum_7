\section{Задание 1}
    Считается доступным лишь генератор равномерно распределенной на 
    отрезке \(\segm{0}{1}\) случайной величины
    $\eta \sim \mathrm{Uni}\inter{0}{1}$. Требуется:

    \begin{enumerate}
        \item Реализовать генератор схемы Бернулли с заданной вероятностью
        успеха $p$. На основе генератора схемы Бернулли построить датчик для
        биномиального распределения.
        \item Реализовать генератор геометрического распределения. Проверить 
        для данного распределения свойство отсутствия памяти.
        \item Рассмотреть игру в орлянку~--- бесконечную последовательность 
        независимых испытаний с бросанием правильной монеты. Выигрыш $S_n$ 
        определяется как сумма по всем $n$ испытаниями $1$ и $-1$ в зависимости 
        от выпавшей стороны. Проиллюстрировать (в виде ломанной) поведение 
        нормированной суммы $Y(i) = S_i/\sqrt{n}$, как функцию от номера 
        испытания $i = 1,\dots, n$ для одной отдельно взятой траектории. Дать 
        теоритическую оценку для $Y(n)$  при $n \rightarrow \infty$.
    \end{enumerate}

    \subsection{Реализация схемы Бернулли и биномиального распределения}
        Чтобы практически реализовать схему Бернулли нужно получить н.о.р.с.в.\\ 
        $\xi_i \sim \mathrm{Bern}(p), i = 1,\dots,n$. Для этого достаточно 
        выразить $\xi_i$ через $\eta$ следующим образом: $\xi_i = 
        \mathbb{I}(\eta < p) + \mathbb{I}(\eta \ge p)$, т.е.

        \[\xi_i = \left\{\begin{aligned}
                            1,&\quad \eta < p,\\
                            0,&\quad \eta \ge p.
        \end{aligned}\right.\]
    
        В свою очередь $\beta \sim \mathrm{Bin}(n,p)$ можно представить как 
        $\beta = \sum\limits_{i=1}^{n} \xi_i$.

        Программа, по описанной выше схеме моделирующая $\mathrm{Bin}(16,0.5)$, 
        дает следующий результат (рис. \ref{task1_bin}).
        \newpage
        \begin{figure}[h]
            \centering
            \includegraphics[width=0.4\textwidth]{resources/task1_bin.png}
            \caption{Биномиальное распределение (размер выборки~--- $1000$)}
            \label{task1_bin}
        \end{figure}

    \subsection{Геометрическое распределение}
        Случайная величина $\gamma \sim \mathrm{Geom}(p)$ представима как 
        $\gamma = \min\{i \in \mathbb{N} : \xi_i = 1\}.$
        \bigskip

        Геометрическое распределение обладает свойством отсутствия памяти, т.е.\\
        $\mathbb{P}(\gamma > m + n \mid \gamma \ge m) = \mathbb{P}(\gamma > n), 
        \forall{m,n} \in \mathbb{N} \cup \{0\}.$