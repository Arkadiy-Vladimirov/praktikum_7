\section{Задание 4}

\begin{enumerate}
    \item Построить датчик распределения Коши. 
    \item На основе датчика распределения Коши с помощью метода фон Неймана построить датчик стандартного нормального распределения. При помощи функции normal probabitity plot убедиться в корректности построенного датчика и обосновать наблюдаемую линейную зависимость.
    \item Сравнить скорость моделирования стандартного нормального распределения в заданях 3 и 4.
\end{enumerate}

\subsection{Датчик распределения Коши}
    Для моделирования абсолютно непрерывных распределений используется метод 
    обратной функции, основанный на том факте, что случайная величина 
    \begin{equation*}
        \xi = F^{-1}(\eta), \text{ где }  \eta \sim \mathrm{U}(0,1) 
    \end{equation*}
    имеет фунцию распределения $F(x)$. ($F(x)$~--- непрерывная, строго 
    возрастающая функция распределения.)

    Функция распределения Коши $F(x) = \frac{1}{2} + 
    \frac{1}{\pi}\arctg(\frac{x-x_0}{\gamma})$. Обратная к ней 
    $F^{-1}(y)~=~x_0+\gamma\tg[\pi(x-\frac{1}{2})]$. Таким образом искомая 
    формула для моделирования $\xi~\sim~\mathrm{Cauchy}(x_0,\gamma)$ имеет вид
    \begin{equation*}
        \xi = x_0+\gamma\tg[\pi(\eta-\frac{1}{2})],\; \eta \sim \mathrm{U}(0,1).
    \end{equation*}
    Пример результата работы полученного датчика представлен на рис.\ref{cauchy}.

    \begin{figure}[tbp]
        \centering
        \includegraphics[width=0.5\textwidth]{resources/task4_cauchy.png}
        \caption{}
        \label{cauchy}
    \end{figure}

\subsection{Датчик стандартного нормального распределения 2}
    Будем генерировать стандартную нормальную случайную величину методом 
    Фон-Неймана используя стандартное распределение Коши 
    ($\eta \sim \mathrm{Cauchy}(0,1)$) и распределение Бернулли.

    Требуемая выборка $\{X_i\}_{i=1}^n$ получается следующим образом. Для очередного $i$
    генерируется некоторое значение $x$ из закона $\eta \sim \mathrm{Cauchy}(0,1)$
    до тех пор, пока результат проведенного затем испытания Бернулли $\nu(x)$ с 
    вероятностью успеха $\frac{\sqrt{e}}{2} e^{-\frac{x^2}{2}} (x^2+1)$ не 
    будет положительным. Тогда значение элемента выборки $X_i$ принимается 
    равным $x$.

    Результат моделирования см. на рис. \ref{norm_neumann}.

    \begin{figure}[tbp]
        \centering
        \includegraphics[width=0.5\textwidth]{resources/task4_norm.png}
        \caption{}
        \label{norm_neumann}
    \end{figure}

\subsection{Сравнение реализаций датчика стандартного нормального распределения}
    